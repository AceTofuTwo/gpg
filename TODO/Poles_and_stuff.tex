\documentclass{article}

\begin{document}


First we begin with the concept of magnetic poles. They can't be generated in practise. If you cut a small magnet in half, you will have two smaller dipole magnets. Let $Q$ be a magnetic charge. It has units of Webers.The charge creates a magnetic field, $B$ that is given by


\begin{equation}
B=  \frac{ \mu_0 Q \hat r}{4\pi r^2}
\end{equation}

If $Q$ is positive the field lines of $B$ extend radially outward in all directions as indicated by the drawing. If $Q$ is negative the field lines have the same shape but they point toward the source. If a positive and negative charge are put in proximity they form a dipole and the field lines look like the diagram below.


\bigskip

If the distance between the two charges is $s$ then the dipole has a magnetic moment $m=Qs$. (units $Amp m^2$) The direction of the dipole goes from the positive to negative charge. (see diagram). Formulae for the magnetic field in cylindrical or cartesian coordinates can be found xxxx.

\bigskip

As an aside we notice that magnetic charges behave exactly as point electric charges. An important distinction is that electric particles can exist by themselves whereas magnetic charges always occur in pairs. The reason for this is that all magnetic fields fundamentally are arise from currents.


\subsection{Working with magnetic charges}

The magnetization in a body of constant magnetic susceptibility $\kappa$ is $\vec M = \kappa \vec H_0$. As illustrated in the above diagram, the magnetic field outside the body can be represented as fields  due to charges on the surface of the body. The surface charge density is given by


\begin{equation}
\tau_s= \vec M \cdot \hat n
\end{equation}

Note:  Possible App. Place a prismatic body in a  constant field and evaluate $\tau_s$ as the body is rotated with respect to the field.

There are some circumstances in which the concept of magnetic charge greatly simplifies the problem. Consider a pipe, or vertical prism, and an incident magnetic field that is pointing down. The magnetization points vertically downward and $\vec{M} \cdot \hat n$ is zero except at the two ends. At the top the charge density is $|M| W/m^2$ and at the bottom it is $-|M| W/m^2$. Suppose the pipe has a radius $a$ and thus an area $\pi a^2$. If the radius of the pipe is small compared to the location of the observer then the effect is the same as if all of the charge was sitting at the top of the pipe at its center. The total charge on the face is the area (units $m2$) times the charge density $W/m^2$

\begin{equation}
Q= \kappa H_0 \pi*a^2
\end{equation}

and the magnetic fields are like those given in equation XXX and diagram XXX.

The same phenomenon is happening at the bottom of the pipe but there the charge is $-Q$. At the surface the magnetic field is the sum of fields due to the two charges, but if the pipe is very long, then the contribution from the bottom of the pipe becomes negligible. The resultant observed field is effectively that due to a monopole, or point charge, of strength Q.  This handy simplification often arises in practise.


\subsection{Magnetic Remanence}
... make alterations to the text but an app be extremely useful.

\end{document}
